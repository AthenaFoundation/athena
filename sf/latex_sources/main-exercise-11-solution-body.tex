% ------------------------------------------------------------------------
% file `main-exercise-11-solution-body.tex'
%
%     solution of type `exercise' with id `11'
%
% generated by the `solution' environment of the
%   `xsim' package v0.16a (2020/01/16)
% from source `main' on 2024/06/09 on line 82
% ------------------------------------------------------------------------
We use structural induction. When \mtt{n} is \mtt{Zero}, we need to show \mtt{(\url{~}$\ls$even Succ Zero)}, which follows
directly from the definition of \mtt{even}.

For the inductive case, assume that \mtt{n} is of the form \mtt{(Succ k)}, so that our inductive hypothesis becomes
\begin{equation}
\mtt{(even k <==> \url{~}$\ls$even Succ k)}
\label{Eq:IndHyp}
\end{equation}
We now need to derive the following conditional:
\begin{equation}
\mtt{(even Succ k <==> \url{~}$\ls$even Succ Succ k)}.
\label{Eq:Goal1}
\end{equation}

In the left-to-right direction, assume \mtt{(even Succ k)}. Then, from the inductive hypothesis~(\ref{Eq:IndHyp})
we infer
\begin{equation}
\mtt{(\tneg even k)}.
\label{Eq:Aux1}
\end{equation}
But, by the definition of \mtt{even}, we have
\begin{equation}
\mtt{(even Succ Succ k <==> even k)},
\label{Eq:EvenDef}
\end{equation}
thus~(\ref{Eq:Aux1}) becomes the desired conclusion \mtt{(\tneg even Succ Succ k)}.

For the right-to-left direction of the goal~(\ref{Eq:Goal1}), assume \mtt{(\tneg even Succ Succ k)}.
By using~(\ref{Eq:EvenDef}) again (the definition of \mtt{even}), this assumption yields
\mtt{(\tneg even k)}, and applying the inductive hypothesis~(\ref{Eq:IndHyp}) to
\mtt{(\tneg even k)} yields the desired conclusion \mtt{(even Succ k)}.
