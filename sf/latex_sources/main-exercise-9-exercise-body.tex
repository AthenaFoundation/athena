% ------------------------------------------------------------------------
% file `main-exercise-9-exercise-body.tex'
%
%     exercise of type `exercise' with id `9'
%
% generated by the `exercise' environment of the
%   `xsim' package v0.16a (2020/01/16)
% from source `main' on 2024/06/09 on line 807
% ------------------------------------------------------------------------
Here is the phrasing of the exercise from SF:
\begin{quote}
Consider a different, more efficient representation of natural numbers using a binary rather than unary system. That is,
instead of saying that each natural number is either zero or the successor of a
natural number, we can say that each binary number is either
\bit
\item zero,
\item twice a binary number, or
\item one more than twice a binary number.
\eit
\begin{alphaEnum}
\item First, write an inductive definition of the type \smtt{Bin} corresponding to this
description of binary numbers.
\item Next, write an increment function incr for binary numbers, and a function
\smtt{bin\_to\_nat} to convert binary numbers to unary numbers.
\item Write five unit tests \smtt{test\_bin\_incr\_1}, \smtt{test\_bin\_incr\_2}, etc. for your increment and binary-to-unary functions. Notice that incrementing a binary
number and then converting it to unary should yield the same result as
first converting it to unary and then incrementing.
\end{alphaEnum}
\end{quote}
Do all of the above in Athena.
